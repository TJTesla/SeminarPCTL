%default language is German. use the second line instead for english settings:
\RequirePackage{ifpdf}
\documentclass{lni}
%\documentclass[english]{lni}

\IfFileExists{latin1.sty}{\usepackage{latin1}}{\usepackage{isolatin1}}

\usepackage{graphicx}
\usepackage[utf8]{inputenc}

\usepackage{latexsym}

\author{
	Theodor Teslia \\ 
	\\ 
	Informatik 11 -- Embedded Software \\ 
	RWTH Aachen University \\ 
	Aachen, Germany \\ 
	teslia@embedded.rwth-aachen.de\\
	\\
	\textit{Betreuer}\\
	\textit{Robin Mross}\\ %Dies ist der inhaltliche Betreuer. Wer das ist, erfahren Sie noch.
}
\title{\small{Proseminar} \\ \vspace{0.5cm} \Large{Eine Logik für das Schlussfolgern über Zeit und Zuverlässigkeit}}

\newcommand{\CTL}{\mathsf{CTL}}
\newcommand{\PCTL}{\mathsf{PCTL}}

\begin{document}
\maketitle

\begin{abstract}
Eine prägnante Zusammenfassung des Kerninhaltes ohne thematische Einleitung und Fazit. 
\end{abstract}

\section{Einführung}

Hier führe ich in die Logik $\PCTL$ ein, die Probleme die diese löst und wie dies circa gemacht wird.
Falls es funktioniert, führe ich außerdem in Halbringsemantik ein und erkläre kurz, in welchem Punkt sich die Ansätze ähneln und unterscheiden.

\section{Grundlagen}

\emph{Noch keine Rückmeldung von Robin.} Hier wird kurz die Syntax von $\CTL$ eingeführt und die Semantik, sowie Modellbeziehung erklärt.

\section{Eine Logik für Zeit und Zuverlässigkeit}

Dieses Kapitel stellt das Hauptkapitel der Arbeit dar und soll die Syntax und Semantik von $\PCTL$ erläutern, sowie mithilfe von eigenen Beispielen diese Verständlicher machen.

\subsection{Syntax}

Hier wird die Syntax von $\PCTL$ erläutert und einige Beispiele, die korrekt bzw. inkorrekt gebildete Formeln gebracht werden.

\subsection{Semantik}

Hier wird die Semantik erklärt und das verwendete Transitionssystem, die Markov Kette eingeführt. 
Einige simple, abstrakte Beispiele ebenfalls zum Erklären verwendet werden.
Ein Vergleich mit $\CTL$ soll passieren, wobei auf die in \emph{Paper} definierte Einbettung verwiesen, aber auch mein Fehler \emph{(falls meine Überlegungen korrekt sind)} gezeigt wird.

\subsection{Model-Checking Algorithmen}

Die im \emph{Paper} genannten Model-Checking Algorithmen werden erklärt und anhand von Tabellen und unterschiedlichen Formeln erläutert.
Wie im \emph{Paper} soll eine Aufteilung in die unterschiedlichen Parameter-\textit{Arten} stattfinden.

\subsection{Angewandtes Beispiel}

Was genau als Beispiel verwendet wird, muss noch entschieden werden, im Zweifelsfall ein etwas anderes Übertragungsprotokoll als im Paper.
Einige unterschiedliche Formeln sollen übersetzt werden so, dass auch die Bedeutung anderer Formeln als nur $\leadsto$ klarer wird.

\section{Vergleich mit Halbringsemantik}

\emph{Ob dieses Kapitel umgesetzt wird hängt von der Vergleichbarkeit von Halbringsemantik+CTL mit $\PCTL$ ab. Der Platz im restlichen Teil der Arbeit, der durch die Existenz des Kapitels wegfällt, soll von Kapitel \ref{ChapVerwandt} gepuffert werden. Bei Platzproblemen kann über das Zusammenlegen der Kapitel \ref{HalbringFO} und \ref{HalbringCTL} nachgedacht werden.} 
Hier soll kurz erklärt werden, was genau Halbringsemantik ist und wofür sie im Allgemeinen verwendet wird.

\subsection{Halbringsemantik zum Auswerten von Wahrscheinlichkeiten}
\label{HalbringFO}

Verwendung des Viterbi- bzw. \L ukasiewicz-Halbrings zum Auswerten von $\mathsf{FO}$ auf Graphen.

\subsection{Halbringsemantik für $\CTL$}
\label{HalbringCTL}

Erweiterung der Halbringsemantik für $\mathsf{FO}$ auf $\CTL$.

\subsection{Vergleich von Halbringsemantik für $\CTL$ mit $\PCTL$}

Unterschiede im Ansatz der beiden Varianten. Falls sich diese einfach beheben lassen, Vergleich in der Nutzung, der Ausdruckskraft etc.

\section{Verwandte Arbeiten}
\label{ChapVerwandt}

Andere Ansätze die entweder nur Zeit oder Wahrscheinlichkeiten zu $\CTL$ hinzufügen sollen hier erläutert werden.

\subsection{Erweiterung von $\CTL$ durch Echtzeit}

Ein weiterer Ansatz zum Ergänzen durch Zeit ist die Benutzung von reellen Werten, im Vergleich zu diskreten. \emph{Baier + Katoen und Paper}
Ein Vergleich mit $\PCTL$ bzgl. der Ausdruckskraft soll folgen, evtl. mit Beschränkung auf $t=\infty$ für $\PCTL$.

\subsection{Erweiterung von $\CTL$ durch Wahrscheinlichkeiten}

Es gibt auch Logiken die nur Wahrscheinlichkeiten hinzufügen. Lassen sich unterschiedliche Ansätze finden? Lösen diese andere Probleme \emph{Baier + Katoen und Paper}

\bibliography{references}

\end{document}
