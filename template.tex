%default language is German. use the second line instead for english settings:
\RequirePackage{ifpdf}
\documentclass{lni}
%\documentclass[english]{lni}

\IfFileExists{latin1.sty}{\usepackage{latin1}}{\usepackage{isolatin1}}

\usepackage{graphicx}
\usepackage[utf8]{inputenc}

\usepackage{amsmath}

\usepackage{latexsym}

\author{
	Theodor Teslia \\ 
	\\ 
	Informatik 11 -- Embedded Software \\ 
	RWTH Aachen University \\ 
	Aachen, Germany \\ 
	teslia@embedded.rwth-aachen.de\\
	\\
	\textit{Betreuer}\\
	\textit{Robin Mross}\\ %Dies ist der inhaltliche Betreuer. Wer das ist, erfahren Sie noch.
}
\title{\small{Proseminar} \\ \vspace{0.5cm} \Large{Eine Logik für das Schlussfolgern über Zeit und Zuverlässigkeit}}

\newcommand{\CTL}{\mathsf{CTL}}
\newcommand{\PCTL}{\mathsf{PCTL}}

\begin{document}
\maketitle

\begin{abstract}
Eine prägnante Zusammenfassung des Kerninhaltes ohne thematische Einleitung und Fazit. 
\end{abstract}

\section{Einführung}

Hier führe ich in die Logik $\PCTL$ ein, die Probleme die diese löst und wie dies circa gemacht wird.
Falls es funktioniert, führe ich außerdem in Halbringsemantik ein und erkläre kurz, in welchem Punkt sich die Ansätze ähneln und unterscheiden.

\section{Grundlagen}

Hier wird kurz die Syntax von $\CTL$, dessen Semantik und die Definition Kripkestrukturen für $\CTL$ eingeführt. Ebenfalls soll kurz erläutert werden, was $\models$ bedeutet. Informationen ziehe ich primär aus \cite{baier2008principles}, da hier die Modellbeziehung und das Thema \textit{Model Checking} aus modernerer Sicht betrachtet wird, und \cite{clarke1981design}, da die in \cite{hansson1994logic} verwendete Notation eher der hier vorgestellten entspricht. Falls sich dies nicht unbedingt anbietet kann die zweite Quelle auch weggelassen werden.

\section{Eine Logik für Zeit und Zuverlässigkeit}

Dieses Kapitel stellt das Hauptkapitel der Arbeit dar und soll die Syntax und Semantik von $\PCTL$ erläutern, sowie mithilfe von eigenen Beispielen diese Verständlicher machen. Offensichtliche verwende ich in dem gesamten Kapitel hauptsächlich \cite{hansson1994logic} für Informationen, Algorithmen und Beweise, die Beispiele werden aber zum Großteil eigene sein.

\subsection{Syntax}

Hier wird die Syntax von $\PCTL$ erläutert und einige Beispiele, die korrekt bzw. inkorrekt gebildete Formeln gebracht werden.

\subsection{Semantik}

Hier wird die Semantik erklärt und das verwendete Transitionssystem, die Markov Kette eingeführt. 
Einige simple, abstrakte Beispiele ebenfalls zum Erklären verwendet werden.
Ein Vergleich mit $\CTL$ soll passieren, wobei auf die in \cite{hansson1994logic} definierte Einbettung verwiesen, aber auch der Fehler bzgl. $\operatorname{EG}\varphi$ gezeigt wird.

\subsection{Model-Checking Algorithmen}

Die in \cite{hansson1994logic} genannten Model-Checking Algorithmen werden erklärt und anhand von Tabellen und unterschiedlichen Formeln erläutert.
Wie im Paper soll eine Aufteilung in die unterschiedlichen Parameter-\textit{Arten} (also $0<t<\infty$, $t=0$, $t=\infty$ und analog für $p$) stattfinden.

\subsection{Angewandtes Beispiel}

Was genau als Beispiel verwendet wird, muss noch entschieden werden, im Zweifelsfall ein etwas anderes Übertragungsprotokoll als im Paper.
Einige unterschiedliche Formeln sollen übersetzt werden so, dass auch die Bedeutung anderer Formeln als nur $\leadsto$ klarer wird.

\section{Vergleich mit Halbringsemantik}

\emph{Ob dieses Kapitel umgesetzt wird hängt von der Vergleichbarkeit von Viterbi-Halbring + $\CTL$ mit $\PCTL$ ab. Der Platz im restlichen Teil der Arbeit, der durch die Existenz des Kapitels wegfällt, soll von Kapitel \ref{ChapVerwandt} gepuffert werden. Bei Platzproblemen kann über das Zusammenlegen der Kapitel \ref{HalbringFO} und \ref{HalbringCTL} nachgedacht werden.}
Hier soll kurz erklärt werden, was genau Halbringsemantik ist und wofür sie im Allgemeinen verwendet wird. In dieser Einleitung und Kapitel \ref{HalbringFO} wird vor allem auf \cite{gradel2017semiring} verwiesen.

\subsection{Halbringsemantik zum Auswerten von Wahrscheinlichkeiten}
\label{HalbringFO}

Verwendung des Viterbi-Halbrings zum Auswerten von $\mathsf{FO}$ auf Graphen. Dieses Kapitel wird bei Platzproblemen weggelassen und der Übergang von $\mathsf{FO}$ auf $\mathsf{CTL}$ wird kurz in Kapitel \ref{HalbringCTL} passieren. Falls ich kein veröffentlichtes Paper mit den Informationen finde, benutze ich \cite{gradel2022provenance}.

\subsection{Halbringsemantik für $\CTL$}
\label{HalbringCTL}

Erweiterung der Halbringsemantik für $\mathsf{FO}$ auf $\CTL$. Informationen sollen aus \cite{dannert2019generalized} und \cite{lluch2005quantitative} stammen.

\subsection{Vergleich von Halbringsemantik für $\CTL$ mit $\PCTL$}

Unterschiede im Ansatz der beiden Varianten. Falls sich diese einfach beheben lassen, Vergleich in der Nutzung, der Ausdruckskraft etc.

\section{Verwandte Arbeiten}
\label{ChapVerwandt}

Andere Ansätze die entweder nur Zeit oder Wahrscheinlichkeiten zu $\CTL$ bzw. Fixpunktlogiken hinzufügen sollen hier erläutert werden.

\subsection{Erweiterung von $\CTL$ durch Echtzeit}

Ein anderer Ansatz zum Ergänzen durch Zeit ist die in \cite{alur1990model} definierte Logik, welche erläutert werden soll. Interessant ist auch die Logik aus \cite{jahanian1986safety}, welche eine Fixpunkt-Logik, also ausdrucksstärker als $\CTL$ ist.
Ein Vergleich mit $\PCTL$ bzgl. der Ausdruckskraft soll folgen, evtl. mit Beschränkung von $p$ auf extreme Werte, also $p\in \{0,1\}$ für $\PCTL$.

\subsection{Erweiterung von $\CTL$ durch Wahrscheinlichkeiten}

Es gibt auch Logiken die nur Wahrscheinlichkeiten hinzufügen. Lassen sich unterschiedliche Ansätze finden? Lösen diese andere Probleme? Paper mit anderen probabilistischen Logiken (die $\CTL$ erweitern): \cite{hart1984probabilistic}, \cite{lehmann1982reasoning} und \cite{christoff1992reasoning}.

\section{Konklusion}

Hier fasse ich noch einmal kurz die Ergebnisse zusammen. Also was $\PCTL$ ist, wofür es benutzt werden kann und evtl. wie Halbringsemantik ein ähnliches Ergebnis erzielen kann.

\bibliography{references}

\end{document}
