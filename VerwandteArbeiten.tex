\section{Verwandte Arbeiten}
\label{ChapVerwandt}

Da nun die Logik $\PCTL$ bekannt ist sollen einige alternative Ansätze bzw. Logiken diskutiert werden, die entweder zeitliche Aspekte oder Wahrscheinlichkeiten hinzufügen.
Zuerst sollen zwei Logiken betrachtet werden, die Zeit implementieren. 
Eine davon unterscheidet sich von $\PCTL$ dadurch, dass Zeit nicht mehr diskret durch Hochzählen von Transitionen behandelt wird, sondern reelle Werte dafür verwendet werden \cite{alur1990model}. 
Die Andere ist eine Verallgemeinerung über Propositionale Temporale Logik $\mathsf{PTL}$ und verwendet einen sogenannten \textit{Einfrier-Operator} \cite{alur1994really}.
Nachdem diese Logiken miteinander und mit $\PCTL$ verglichen wurden, sollen noch Wahrscheinlichkeiten betrachtet werden.
Dafür werden Logiken vorgeführt, welche es ermöglichen Wahrscheinlichkeiten auszudrücken.
% TODO

% Andere Ansätze die entweder nur Zeit oder Wahrscheinlichkeiten zu $\CTL$ bzw. Fixpunktlogiken hinzufügen sollen hier erläutert werden.

\subsection{Logiken mit Echtzeit}

In diesem Kapitel werden die Logiken \textit{Timed Computation Tree Logic} ($\TCTL$) und \textit{Timed Propositional Temporal Logic} ($\TPTL$) anhand ihrer Syntax sowie Semantik erörtert und ihre Model-Checking Möglichkeit diskutiert. 
Zusätzlich werden diese dann jeweils mit $\PCTL$ verglichen.

\subsubsection{Timed Computation Tree Logic}

Die Logik $\TCTL$ stellt, ähnlich wie $\PCTL$, eine Möglichkeit zur Verfügung, um zeitliche Zusammenhänge auszudrücken.
Während dies in $\PCTL$ aber nur mit diskreten Werten möglich ist, verwendet $\TCTL$ reelle Werte, wodurch sich Systeme potentiell besser beschreiben lassen.
Da dies aber offensichtlicher weise komplexer, als bloß bei jeder verwendeten Transition einen Zähler zu erhöhen, wird eine neue Struktur benötigt, welche Systeme modellieren soll und über welchen $\TCTL$-Formeln ausgewertet werden.
Dafür definieren wir eine feste Menge $\mathcal{N}$, welche uns Werte für zeitliche Vergleiche in Formeln gibt.
Der Einfachheit halber definieren wir hier $\mathcal{N}\coloneqq\{0,1,\dots\}=\mathbb{N}$.
Da es aber offensichtlich eine Bijektion zwischen $\mathbb{N}$ und $\mathbb{Q}$ gibt, lässt sich auch $\mathcal{N}\coloneqq\mathbb{Q}$ festlegen, wodurch sich die Vorteile einer dichten Ordnung\footnote{Eine dichte Ordnung ist eine lineare Ordnung so, dass zwischen je zwei Elementen ein drittes liegt.} im Vergleich zu $\PCTL$ ausnutzen lassen.

\begin{definition}[Zeitliche Graphen]
	Ein \textit{zeitlicher Graph} ist ein Tupel $\mathfrak{S}=(S,s_0,E,C,\pi,\tau,\mu)$ mit folgenden Definitionen:
	\begin{itemize}
		\item $S$ ist eine endliche Menge an Knoten.
		\item $s_0\in S$ ist der Startzustand.
		\item $E\subseteq S\times S$ ist die Kantenrelation.
		\item $C$ ist eine endliche Menge an Uhren.
		\item $\pi:E\to 2^C$ ist eine Abbildung, die jeder Kante eine Menge an Uhren zuweist.
		\item $\tau$ ist eine Abbildung, die jeder Kante eine Formel zuweist, die aus Booleschen Junktoren über den atomaren Formeln $x\leq c$ und $c\leq x$ besteht, wobei $c\in C$ und $x\in \mathcal{N}$ gilt.
		\item Für eine Menge an atomaren Aussagen $\mathsf{AP}$ ist $\mu:S\to 2^{\mathsf{AP}}$ eine Abbildung, die jedem Zustand eine Menge an atomaren Aussagen zuweist.
	\end{itemize}
	
	Ein zeitlicher Graph besitzt also eine Menge an sogenannten Uhren $C$.
	Diese zählen unabhängig voneinander hoch so, dass jede Uhr zu jedem \glqq Zeitpunkt\grqq{} einen reellen Wert $x\in \mathbb{R}_{\geq 0}$ speichert.
	Mithilfe von $\pi$ lassen sich Uhren zurücksetzen, für jede Transition lässt sich also eine Menge an Uhren auswählen, welche mit dem Wechsel über die Transition zurückgesetzt werden.
	Zusätzlich existiert für jede Kante eine Formel, welche bestimmte Voraussetzungen an die Transition stellt.
\end{definition}

Um diese Definition besser zu verstehen, soll ein Beispiel angeführt werden. Sei 
\begin{align*}
\mathfrak{S}&=( \\
&S= \{s_0,s_1,s_2\},\\
&s_0,\\
&E=\{(s_0,s_2),(s_1,s_0),(s_1,s_1),(s_1,s_2),(s_2,s_1)\},\\
&C=\{x,y\},\\
&\pi(u,v) = \begin{cases}
	\{x\} & \text{falls } v = s_2 \\
	\{y\} & \text{falls } u = s_2 \\
	\emptyset & \text{sonst}
\end{cases}\\
&\tau=\{\}\\
)&
\end{align*}
Dieses lässt sich in Abbildung \ref{} sehen.

\begin{figure}[h]
	\begin{center}
		\begin{tikzpicture}[shorten >= 1pt, node distance=6cm, on grid, auto]
			\node[state, initial above, initial text={}, align=center] (s_0) [] {$s_0$ \\ $\{\mathsf{running}\}$};
			\node[state, align=center] (s_1) [right=of s_0] {$s_1$ \\ $\{\mathsf{stopped},$\\$\mathsf{warning}\}$};
			\node[state, align=center] (s_2) [below=of s_1] {$s_2$ \\ $\{\mathsf{stopped},$\\$\mathsf{error}\}$};
			
			\path[-stealth]
			(s_0) edge [] node[align=center] {$\top$ \\ $\reset{x}$} (s_2)
			
			(s_1) edge [bend right] node {$0.1$} (s_2)\textsc{}
			(s_1) edge [] node {$0.4$} (s_0)
			
			(s_2) edge [bend right] node {$0.4$} (s_1)
			;
		\end{tikzpicture}
		
		\caption{Ein Beispiel für einen zeitlichen Graphen}
		\label{ZeitlicherGraph1}
	\end{center}
\end{figure}

% Ein anderer Ansatz zum Ergänzen durch Zeit ist die in \cite{alur1990model} definierte Logik, welche erläutert werden soll. Interessant ist auch die Logik aus \cite{jahanian1986safety}, welche eine Fixpunkt-Logik, also ausdrucksstärker als $\CTL$ ist.
% Ein Vergleich mit $\PCTL$ bzgl. der Ausdruckskraft soll folgen, evtl. mit Beschränkung von $p$ auf extreme Werte, also $p\in \{0,1\}$ für $\PCTL$.

\subsection{Erweiterung von $\CTL$ durch Wahrscheinlichkeiten}

Es gibt auch Logiken die nur Wahrscheinlichkeiten hinzufügen. Lassen sich unterschiedliche Ansätze finden? Lösen diese andere Probleme? Paper mit anderen probabilistischen Logiken (die $\CTL$ erweitern): \cite{hart1984probabilistic}, \cite{lehmann1982reasoning} und \cite{christoff1992reasoning}.