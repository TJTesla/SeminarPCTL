\section{Konklusion}

In dieser Arbeit wurde sich eingehend mit der Logik $\PCTL$ beschäftigt.
Diese stellt eine Erweiterung der Logik $\CTL$ dar und ermöglicht es, Eigenschaften von Soft-Realtime-Systemen auszudrücken, da sie sowohl über Wahrscheinlichkeiten, als auch Zeit Aussagen treffen kann.
Erzielt wird dies durch einen erweiterten \textit{Until}-Operator, welcher einen Zeit und einen Wahrscheinlichkeitsparameter erhält.
Insgesamt ergeben sich sehr effiziente Model-Checking Algorithmen für unterschiedliche Fälle.

Vergleicht man $\PCTL$ mit anderen Logiken fällt auf, dass diese Logik eine sehr hohe Ausdruckskraft bzgl. Wahrscheinlichkeiten besitzt.
So sind die dafür verwendeten Mechanismen die gleichen wie für $\GPL$, welche eine deutlich stärkere Logik darstellt und eine Vielzahl an anderen Logiken ausdrücken kann.
Bzgl. Zeit existieren aber sehr viel mächtigere Implementationen, wie am Beispiel von $\TCTL$ gesehen wurde, welche sehr viel mehr zeitliche Aspekte formulieren kann als $\PCTL$.

% Hier fasse ich noch einmal kurz die Ergebnisse zusammen. Also was $\PCTL$ ist, wofür es benutzt werden kann und evtl. wie Halbringsemantik ein ähnliches Ergebnis erzielen kann.