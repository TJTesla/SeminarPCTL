\section{Vergleich mit Halbringsemantik}

Ein weiterer Ansatz um die Ausdrucksstärke einer Logik zu erweitern ist es, Formeln nicht nur zu \textit{wahr} oder \textit{falsch} auszuwerten, sondern einer Formel einen Wert aus einem Halbring zuzuordnen.
Diese Methode wird als Halbringsemantik bezeichnet.
Abhängig davon, in welcher Halbring verwendet wird, lassen sich unterschiedliche Anwendungen darstellen. 
So gibt es zum Beispiel den natürlichen Halbring $\mathbb{N}=(\mathbb{N},+,\cdot,0,1)$ zum Zählen, den tropischen Halbring $\mathbb{T}=(\mathbb{R}^\infty_+\coloneqq\mathbb{R} \cup \{0,\infty\},\min,+,\infty,0)$ für Kostenanalyse oder den Viterbi-Halbring $\mathbb{V}=\{[0,1],\max,\cdot,0,1\}$ für Vetrauensanalyse \cite{dannert2019provenance}.
Letzterer soll im Weiteren behandelt und, verbunden mit $\CTL$ im Vergleich zu $\PCTL$ betrachtet werden.
Dafür wird nun aber erst die Halbringsemantik für die Prädikatenlogik erster Ordnung ($\FO$) besprochen.

\subsection{Halbringsemantik für $\FO$}
\label{ChapHalbringFO}

Die Syntax als auch Semantik von $\FO$ wird als bekannt vorausgesetzt.
Falls dies nicht der Fall ist, lassen sich die wichtigen Informationen in \cite{barwise1977introduction} oder \cite{ebbinghaus1994mathematical} nachlesen.
Um die hier verwendete Notation zu erklären soll $\FO$ aber dennoch definiert werden.

\begin{definition}[Syntax von $\FO$]
	Eine Signatur $\tau$ ist eine Menge an Relationen und Funktionen fester Stelligkeit.
	Dann ist die Menge der $\tau$-Terme $\operatorname{Terms}(\tau)$ definiert durch
	\begin{itemize}
		\item Wenn $x$ eine Variable ist, dann ist $x\in\operatorname{Terms}(\tau)$
		\item Wenn $a$ eine Funktion mit Stelligkeit $0$ ist (also eine Konstante), dann gilt $a\in \operatorname{Terms}(\tau)$
		\item Wenn $t_1,\dots,t_k$ $\tau$-Terme sind und $f$ eine $k$-stellige Funktion ist, dann gilt $f(t_1\dots t_k)\in \operatorname{Terms}(\tau)$.
	\end{itemize}
	Damit lässt sich nun die Menge der $\FO$-Formeln definieren
	\begin{enumerate}
		\item Wenn $t_1,t_2\in \operatorname{Terms}(\tau)$, dann ist $t_1=t_2\in\FO$ \label{TermEquiv}
		\item Wenn $t_1,\dots,t_k\in \operatorname{Terms}(\tau)$, und $R$ ist eine $k$-stellige Relation in $\tau$, dann ist $R(t_1\dots t_k) \in \FO$ \label{RelationSyntax}
		\item Wenn $\varphi\in \FO$, dann ist $\neg\varphi\in \FO$
		\item Wenn $\varphi_1,\varphi_2\in \FO$, dann ist $\varphi_1\land\varphi_2,\varphi_1\lor\varphi_2,\varphi_1\rightarrow\varphi_2\in \FO$
		\item Wenn $x$ eine Variable ist und $\varphi\in \FO$, dann ist $\exists x \varphi(x)\in \FO$ und $\forall x \varphi(x)\in \FO$
	\end{enumerate}
	
	Formeln, die nur durch die Regeln \ref{TermEquiv} ud \ref{RelationSyntax} verwenden bezeichnen wir als atomar.
\end{definition}

In diesem Kapitel soll die Halbringsemantik für $\FO$ betrachtet werden. 
Wie der Name schon sagt, wird eine alternative Semantik für $\FO$ verwendet werden.
Aus diesem Grund soll die klassische Semantik hier nicht behandelt werden, Informationen dazu lassen sich ebenfalls in \cite{barwise1977introduction} und \cite{ebbinghaus1994mathematical} finden.

Bevor wir nun aber Halbringsemantik betrachten soll erklärt werden, was ein Halbring ist. 
Ein (kommutativer) Halbring ist eine Struktur $\mathfrak{S}=(S,+,\cdot,0,1)$, wobei $0\neq 1$, $(S,+,0)$, als auch $(S,\cdot,1)$ (abelsche) Monoide sind, $\cdot$ ist distributiv über $+$ und $0\cdot a = a \cdot 0 = 0$ für alle $a\in S$. \cite{dannert2019provenance}
% TODO: Weitere Eigenschaften (+-positivität, Idempotenz etc.) falls im Weiteren benötigt

Jetzt können wir die Halbringsemantik für $\FO$ definieren.
Sei $A$ ein Universum und $\tau$ eine relationale Signatur.
Dann ist $\operatorname{Facts}_A(\tau)\coloneqq\{Ra : R\in\tau, a\in A^{\operatorname{st}(R)}\}$\footnote{Für eine Relation $R$ ist $\operatorname{st}(R)\in \mathbb{N}$ die Stelligkeit von $R$.} und $\operatorname{Lit}_A(\tau)\coloneqq \{\alpha,\neg\alpha : \alpha\in \operatorname{Facts}_A(\tau)\}$.
Sei $\mathfrak{S}$ nun ein kommutativer Halbring. 
Damit können wir eine $\mathfrak{S}$-Interpretation $\pi:\operatorname{Lit}_A(\tau)\to S$ definieren, welche jeder atomaren und negiert-atomaren Formel einen Wert im Halbring zuweist. \cite{dannert2019provenance}

\begin{definition}[Halbringsemantik $\FO$]
	Sei $A$ ein Universum, $\tau$ eine Signatur und $\pi$ eine $\mathfrak{S}$-Interpretation für einen Halbring $\mathfrak{S}$. Dann erweitert sich $\pi$ zu einer Interpretation $\pi:\FO\to S$ wie folgt \cite{dannert2019provenance}
	\begin{itemize}
		\item Für eine Formel $\varphi$ schreiben wir $\operatorname{nnf}(\varphi)$ für die Negations-Normalform von $\varphi$. Dann gilt $\pi \llbracket \neg\varphi\rrbracket = \pi\llbracket \operatorname{nnf}(\neg\varphi)\rrbracket$
		\item $\pi\llbracket \varphi_1\land \varphi_2\rrbracket = \pi\llbracket \varphi_1 \rrbracket \cdot \pi\llbracket \varphi_2 \rrbracket$
		\item $\pi \llbracket \varphi_1 \lor \varphi \rrbracket = \pi\llbracket \varphi_1 \rrbracket + \pi\llbracket \varphi_2 \rrbracket$
		\item $\pi \llbracket \exists x \varphi(x) \rrbracket = \pi \llbracket \sum_{a\in A} \varphi(a) \rrbracket$
		\item $\pi \llbracket \forall x \varphi(x) \rrbracket = \pi \llbracket \prod_{a\in A} \varphi(a) \rrbracket$
	\end{itemize}
\end{definition}

Es lässt sich erkennen, dass $\lor$ und $\exists$ durch $+$, und $\land$ und $\forall$ durch $\cdot$ dargestellt werden.
Informell stellt $+$ also eine alternative Nutzung von Informationen dar, da bei $\lor$ nur eine der beiden Optionen benötigt wird, während $\cdot$ für gemeinsame Nutzung steht, da bei $\land$ alle Optionen gelten müssen.
Um diese Definition zu verdeutlichen soll ein Beispiel vorgeführt werden.

\begin{example}[Beispiel für $\FO$ mit dem Viterbi-Halbring]
	Sei $A=\{1,2,3,4\}$, $\tau=\{E\}$, wobei $E$ ein $2$-stelliges Relationssymbol ist und $\pi: \operatorname{Lit}_A(\tau) \to [0,1]$ eine $\mathbb{V}$-Interpretation in den Viterbi-Halbring $\mathbb{V}=([0,1],\max,\cdot, 0, 1)$ gegeben durch Tabelle \ref{SemiringSemanticsFOBsp}.
	
	\begin{table}[h]
		\begin{center}
			\begin{tabular}{c|cccc}
				$E(x,y)$ & $y=1$ & $y=2$ & $y=3$ & $y=4$ \\
				\hline
				$x=1$ & $0$ & $0.15$ & $0.32$ & $0$ \\
				$x=2$ & $0$ & $0$ & $0$ & $0$ \\
				$x=3$ & $0$ & $0$ & $0$ & $0.21$ \\
				$x=4$ & $0.99$ & $0$ & $0$ & $0.06$ \\
			\end{tabular}
		\end{center}
		\begin{center}
			\begin{tabular}{c|cccc}
				$\neg E(x,y)$ & $y=1$ & $y=2$ & $y=3$ & $y=4$ \\
				\hline
				$x=1$ & $1$ & $0$ & $0$ & $1$ \\
				$x=2$ & $1$ & $1$ & $1$ & $1$ \\
				$x=3$ & $1$ & $1$ & $1$ & $0$ \\
				$x=4$ & $0$ & $1$ & $1$ & $0$ \\
			\end{tabular}
		\end{center}
		\caption{Tabelle zur Definition der $\mathbb{V}$-Interpretation}
		\label{SemiringSemanticsFOBsp}
	\end{table}
	
	Wenn man sich diese Definition anschaut fällt einem auf, dass diese einen Graphen definiert, auf dem jede Kante durch eine Wahrscheinlichkeit annotiert ist.
	Dies liegt daran, dass für jedes $(x,y)\in A^2$ gilt, dass $\pi\llbracket E(x,y)\rrbracket=0$ genau dann, wenn $\pi\llbracket \neg E(x,y) \rrbracket\neq 0$, diese Eigenschaft nennt man auch \textit{modelldefinierend} \cite{dannert2019provenance}.
	Der zugehörige Graph $G$ ist in Abbildung \ref{SemiringSemanticsFOBspGraph} einsehbar.
	
	\begin{figure}[h]
		\begin{center}
			\begin{tikzpicture}[shorten >= 1pt, node distance=4cm, on grid, auto]
				\node[state, initial above, initial text={}, align=center] (1) [] {$1$};
				\node[state, align=center] (2) [right=of 1] {$2$};
				\node[state, align=center] (3) [below=of 1] {$3$};
				\node[state, align=center] (4) [right=of 3] {$4$};
				
				\path[-stealth]
				(1) edge [] node {$0.15$} (2)
				(1) edge [] node {$0.32$} (3)
				
				(3) edge [] node {$0.21$} (4)
				
				(4) edge [] node {$0.99$} (1)
				(4) edge [loop right] node {$0.06$} (4)
				;
			\end{tikzpicture}
			
			\caption{Graph $G$ für die den durch $\pi$ definierten Graphen}
			\label{SemiringSemanticsFOBspGraph}
		\end{center}
	\end{figure}
	
	Eine mögliche Interpretation des Viterbi-Halbrings ist es, diese als Vertrauenswerte zu verwenden. 
	Ein Wert im Viterbi-Halbring gibt also Informationen darüber, wie sicher man über die Korrektheit eines Faktums ist. 
	Hier am Beispiel heißt das zum Beispiel, dass die Kante von Knoten $1$ zu $2$ mit einer Wahrscheinlichkeit von $\pi\llbracket E(1,2)\rrbracket = 0.15$ existiert und mit einer Wahrscheinlichkeit von $\pi\llbracket \neg E(1,2)\rrbracket = 0$ nicht existiert.
	
	Betrachten wir bspw. die Formel $\varphi=\forall x \exists y (Exy)$ die aussagt, dass jeder Knoten eine ausgehende Kante besitzt. 
	Wenn wir also wissen wollen, wie sicher man sich sein kann, dass Graph $G$ diese Eigenschaft besitzt, müssen wir bloß $\pi\llbracket \varphi \rrbracket$ auswerten. Es gilt
	\begin{align*}
		\pi\llbracket \forall x \exists y E(x,y)\rrbracket &= \prod_{a\in A}\pi\llbracket \exists y E(a,y)\rrbracket \\
		&= \prod_{a\in A}\sum_{b\in A} \pi\llbracket E(a,b)\rrbracket \\
		&= \left(\prod_{a\in A\setminus\{2\}}\sum_{b\in A} \pi\llbracket E(a,b)\rrbracket\right) \cdot \underbrace{\left(\sum_{b\in A} \pi\llbracket E(2,b)\rrbracket\right)}_{=0} \\
		&= 0.
	\end{align*}
	Die Formel gilt also nicht für den durch $\pi$ definierten Graphen.
	
	Für die Formel $\psi= \forall y \exists x E(x,y)$ die besagt, dass jeder Knoten eine eingehende Kante hat lässt sich eine ähnliche Rechnung durchführen und man erhält $\pi\llbracket \psi \rrbracket = 0.99\cdot 0.15\cdot 0.32 \cdot 0.21 \approx 1\%$
\end{example}

Somit können wir also ähnliche Aussagen treffen wie dies mit $\PCTL$ und $t=\infty$ möglich ist.
Das Problem ist jedoch, dass $\FO$ auf Graphen sehr wenig Aussagekraft besitzt, weshalb sich diese nicht zur Verifikation anbietet.
Aus diesem Grund soll nun Halbringsemantik für $\CTL$ eingeführt werden.

%Verwendung des Viterbi-Halbrings zum Auswerten von $\mathsf{FO}$ auf Graphen. Dieses Kapitel wird bei Platzproblemen weggelassen und der Übergang von $\mathsf{FO}$ auf $\mathsf{CTL}$ wird kurz in Kapitel \ref{HalbringCTL} passieren. Falls ich kein veröffentlichtes Paper mit den Informationen finde, benutze ich \cite{gradel2022provenance}.

\subsection{Halbringsemantik für $\CTL$}
\label{ChapHalbringCTL}

In diesem Kapitel wollen wir die im letzten Kapitel für $\FO$ entwickelte Semantik auf $\CTL$ ausweiten. 
Die so erhaltene Logik bezeichnen wir als $\cCTL$, was für \textit{constraint-semiring-}$\CTL$ steht \cite{lluch2005quantitative}.
Ein wichtiger Unterschied zu $\FO$ ist aber, dass $\CTL$ eine modale Logik ist, wodurch es nicht ausreicht eine Formel in einen einzelnen Wert eines Halbrings auszuwerten.
Aus diesem Grund definiert eine Formel eine Abbildung von den Kanten des Transitionssystems (siehe Definition \ref{DefKripkeStruk}) in einen Halbring.
Formal definiert ein Halbring $\mathfrak{A}=(A,+,\cdot,0,1)$ und eine Kripkestruktur $\mathfrak{S}=(S, E, \mathcal{L})$ also eine Abbildung $v:E\to A$.
Weiter definieren wir $V\coloneqq A^{E}$ die Menge aller Bewertungen der Zustände.
Damit erhalten wir einen weiteren Halbring $\mathcal{V}=(V,\oplus,\otimes,\boldsymbol{0},\boldsymbol{1})$, mit folgenden Definition:
\begin{itemize}
	\item $\oplus$ ist eine zweistellige Funktion mit $(v\oplus v')(e) = v(e)+v'(e)$
	\item $\otimes$ ist eine zweistellige Funktion mit $(v\otimes v')(e) = v(e)\cdot v'(e)$
	\item $\boldsymbol{0}$ ist eine Konstante für die gilt $\boldsymbol{0}(e) = 0\in A$
	\item $\boldsymbol{1}$ ist eine Konstante für die gilt $\boldsymbol{1}(e) = 1\in A$
\end{itemize}s

\begin{definition}[Syntax von $\cCTL$]
	Sei $\mathfrak{A}=(A+,\cdot,0,1)$ ein wie in Kapitel \ref{ChapHalbringFO} definierter Halbring und sei $a\in A$ und $x\in \mathsf{AP}$. Dann lässt sich $\cCTL$ mithilfe folgender Grammatiken definieren:
	\begin{align*}
		\varphi &\Coloneqq a \, | \, x \, | \, \varphi \land \varphi \, | \, \varphi \lor \varphi \, | \, \neg\varphi \, | \, \kappa \psi \, | \, \kappa \boldsymbol{X}\varphi \\
		\kappa &\Coloneqq \bigsqcup \, | \, \bigsqcap \\
		\psi &\Coloneqq \boldsymbol{F}\varphi \, | \, \boldsymbol{G}\varphi \, | \, [\varphi\boldsymbol{U}\varphi] \, | \, [\varphi\boldsymbol{W}\varphi]
	\end{align*}
	
	Dann ist die Menge der $\cCTL$-Formeln die Ausdrücke, die von $\varphi$ erzeugt werden. \cite{lluch2005quantitative}
\end{definition}

Die intuitiven Bedeutungen von $\land$, $\lor$ und $\neg$ sind wie bekannt.
$a$ stellt eine konstante Bewertung aller Kanten dar, während $v$, analog zu booleschen atomaren Aussagen, jeder Kante einen Halbring-Wert zuweist.
Die durch $\kappa$ erzeugten Ausdrücke sind analog zu Quantoren und die durch $\psi$ erzeugten Ausdrücke stellen die aus $\CTL$ bekannten Begriffe \textit{finally} ($\boldsymbol{F}$), \textit{globally} ($\boldsymbol{G}$), \textit{until} ($\boldsymbol{U}$) und \textit{weak until} ($\boldsymbol{W}$) dar.

Bevor wir diese Bedeutungen formal definieren, soll noch über Negation gesprochen werden.
Im Booleschen-Halbring ist es offensichtlich möglich einen Wert zu invertieren bzw. zu negieren, da es lediglich zwei mögliche Werte gibt und demnach immer \textit{der andere} Wert eindeutig definiert ist.
Im Allgemeinen ist dies aber nicht so trivial. Zum Lösen dieses Problems existieren zwei verschiedene Ansätze. 
Zum Einen ist es möglich einen Negationsoperator $-:A\to A$ zu definieren. 
Zum Anderen lässt sich $v$ ausweiten auf eine Bewertung $v^*:\{E(x,y), \neg E(x,y) : x,y\in S\}\to A$ so, dass, wie in Kapitel \ref{ChapHalbringFO}, nicht nur der Existenz einer Kante ein Wert zugeordnet wird, sondern auch dem Fehlen der Kante.
Im Weiteren soll der Letztere Ansatz verwendet werden.
Demnach definieren wir die Menge $V^*\coloneqq A^{\{E(x,y), \neg E(x,y) : x,y\in S\}}$ aller erweiterten Bewertungen.
Wie für $\FO$ wird demnach dann aber auch eine Negationsnormalform benötigt, die

Zudem betrachten wir $\mathcal{L}$ nicht mehr als Abbildung von Zuständen auf Teilmengen von $\mathsf{AP}$, sondern als Menge $\{v_x : x\in \mathsf{AP}\}$ von Abbildungen $v_x:S\to A$, die jedem Zustand einen Halbring-Wert zuweisen.
Betrachtet man dies mit $\mathfrak{A}=\mathbb{B}$ dem Booleschen-Halbring, dann erhält man $v_x:S\to \{0,1\}$, was uns die gewohnte Bedeutung von atomaren Aussagen gibt.

\begin{definition}[Semantik von $\cCTL$]
	Für einen Halbring $\mathfrak{A}=(A,+,\cdot,0,1)$ und eine Kripke-Struktur $\mathfrak{S}=(S,E,\mathcal{L})$ wird eine $\cCTL$-Formel $\varphi$ zu einer Abbildung $v\in V$ ausgewertet.
	Diese Auswertung lässt sich induktiv mithilfe folgender Regeln definieren \cite{lluch2005quantitative}:
	\begin{itemize}
		\item $\llbracket a \rrbracket(s) = a$ und $\llbracket v \rrbracket(s) = v(s)$ für alle $s\in S$
		\item $\llbracket \varphi_1 \land \varphi_2 \rrbracket = \llbracket\varphi_1\rrbracket \otimes \llbracket\varphi_2\rrbracket$, $\llbracket \varphi_1\lor\varphi_2 \rrbracket = \llbracket \varphi_1\rrbracket \oplus \llbracket \varphi_2\rrbracket$, $\llbracket \neg \varphi \rrbracket = -\llbracket \varphi \rrbracket$
		\item $\llbracket $
	\end{itemize}
\end{definition}


% Erweiterung der Halbringsemantik für $\mathsf{FO}$ auf $\CTL$. Informationen sollen aus \cite{dannert2019generalized} und \cite{lluch2005quantitative} stammen.

\subsection{Vergleich von Halbringsemantik für $\CTL$ mit $\PCTL$}

Unterschiede im Ansatz der beiden Varianten. Falls sich diese einfach beheben lassen, Vergleich in der Nutzung, der Ausdruckskraft etc.