\section{Grundlagen}
\label{ChapGrundlagen}

Um Erweiterungen einer Logik zu verstehen ist es ratsam, die Grundlegende auch zu kennen. Aus diesem Grund soll in diesem Kapitel in die Logik \textit{Computation Tree Logic} ($\CTL$) und naheliegende, wichtige Themen eingeleitet werden.

$\CTL$ ist eine temporale Logik um Aussagen in nicht-deterministischen Systemen zu treffen. 
Dafür betrachtet man Kripkestrukturen, welche eine Art Graph darstellen, da diese viele Eigenschaften liefern, die für $\CTL$-Anwendungsfälle interessant sind.

\begin{definition}[Syntax von $\CTL$]
	Die Menge der $\CTL$-Formeln lässt sich induktiv mithilfe folgender Regeln definieren \cite{clarke1982design, baier2008principles}:
	\begin{enumerate}
		\item Wenn $\mathsf{AP}$ eine Menge an atomaren Aussagen ist, dann ist jedes $a\in \mathsf{AP}$ eine $\CTL$-Formel. Außerdem sind $\top$ und $\bot$ $\CTL$-Formeln.
		\item Wenn $\varphi_1$ und $\varphi_2$ $\CTL$-Formeln sind, dann sind $\neg\varphi_1$, $\varphi_1 \land \varphi_2$ und $\varphi_1 \lor \varphi_2$ ebenfalls $\CTL$-Formeln.
		\item Wenn $\varphi_1$ eine $\CTL$-Formel ist, dann sind auch $\operatorname{EX}\varphi_1$ und $\operatorname{AX}\varphi_1$ $\CTL$-Formeln.
		\item Wenn $\varphi_1$ und $\varphi_2$ $\CTL$-Formeln sind, dann sind $\EU{\varphi_1}{\varphi_2}$, $\AU{\varphi_1}{\varphi_2}$, $\EW{\varphi_1}{\varphi_2}$ und $\AW{\varphi_1}{\varphi_2}$ auch $\CTL$-Formeln.
	\end{enumerate}
\end{definition}

Bevor die Semantik erläutert wird, sollen zuerst die Strukturen gezeigt werden, die man typischerweise mit $\CTL$ zusammen verwendet.

\begin{definition}[Kripkestrukturen]
	\label{DefKripkeStruk}
	Eine \textit{Kripkestruktur} oder \textit{Transitionssystem} ist ein Graph von der Form $\mathcal{K}=(V, E, \mathcal{L})$, wobei $E$ eine zweistellige Kantenrelation über $V$ ist und $\mathcal{L}$ eine Funktion $\mathcal{L}:V\to2^{\mathsf{AP}}$ ist, die jedem Zustand eine Menge an atomaren Aussagen zuweist 	\cite{clarke1982design,clarke1986automatic}.
	
	Weiter bezeichnen wir einen Pfad als $\sigma=s_0 s_1 \dots$ mit $s_i\in V$ für alle $i\in\{0, 1,\dots,\}$ und $\sigma[i]=s_i$ für $i\in \mathbb{N}$, bzw. für endliche Pfade $\sigma = s_0 \dots s_n$ und $\sigma[i] = s_i$ für $0\leq i \leq n$ \cite{baier2008principles}.
\end{definition}

\begin{definition}[Semantik von $\CTL$]
	Für ein Transitionssystem $\mathcal{K}=(V,E,\mathcal{L})$ können wir damit induktiv die Modellbeziehung $\models$ zwischen einem Knoten $v\in V$ und $\CTL$-Formeln definieren \cite{baier2008principles}:
	\begin{enumerate}
		\item $v\models \top \Leftrightarrow \mathcal{K}\models\forall x(x=x)$ und $v\models \bot \Leftrightarrow \mathcal{K}\models\exists x(x\neq x)$.
		\item Für $a\in \operatorname{AP}$ gilt $v\models a \Leftrightarrow a\in \mathcal{L}(v)$.
		\item Es gilt $v\models \neg \varphi \Leftrightarrow v\not\models \varphi$,\\
		$v\models \varphi_1 \land \varphi_2 \Leftrightarrow v\models \varphi_1 \text{ und } v\models \varphi_2$,\\
		$v\models \varphi_1 \lor \varphi_2 \Leftrightarrow v\models \varphi_1 \text{ oder } v\models \varphi_2$,
		\item Es gilt $v\models \operatorname{EX}\varphi \Leftrightarrow$ es ex. ein $w\in V$ mit $(v,w)\in E$ und $w\models \varphi$ und analog:\\ 
		$v\models \operatorname{AX}\varphi \Leftrightarrow$ für alle $w\in V$ mit $(v,w)\in E$ gilt $w\models \varphi$.
		\item Es gilt $v\models \EU{\varphi_1}{\varphi_2} \Leftrightarrow$ es existiert ein Pfad $\sigma$ in $\mathcal{K}$, der in $v$ beginnt und ein $i\in \mathbb{N}$ so, dass $\sigma[i]\models \varphi_2$ und für alle $0\leq j < i$ gilt $\sigma[i]\models \varphi_1$. 
		Analog ist die Definition für $\AU{\varphi_1}{\varphi_2}$, es muss dann aber für jeden in $v$ beginnenden Pfad ein $i\in \mathbb{N}$ geben so, dass der $i$-te Zustand $\varphi_2$ und alle Zustände davor $\varphi_1$ erfüllen.
		\item Es gilt $v\models \EW{\varphi_1}{\varphi_2} \Leftrightarrow$ es existiert ein (unendlicher) Pfad $\sigma$ so, dass entweder ein $i\in \mathbb{N}$ existiert mit $\sigma[i]\models \varphi_2$ und für alle $0\leq j < i$ gilt $\sigma[j]\models \varphi_1$, oder es gilt $\sigma[i]\models \varphi_1$ für alle $i\in \mathbb{N}$.
		Für $\AW{\varphi_1}{\varphi_2}$ ist die Definition analog, bloß mit einem All- anstatt Existenzquantor.
	\end{enumerate}
\end{definition}

Intuitiv bedeutet $\operatorname{EX}\varphi$ also, dass $\varphi$ in einem beliebigen Nachfolgezustand gelten muss und $\operatorname{AX}\varphi$, dass $\varphi$ von allen Nachfolgezuständen erfüllt wird.

$\EU{\varphi_1}{\varphi_2}$ bzw. $\AU{\varphi_1}{\varphi_2}$ sagt aus, dass es einen Pfad $\sigma$ gibt bzw. auf allen Pfaden $\sigma$ gilt, dass zuerst $\varphi_1$ gilt, bis von einem Zustand $\varphi_2$ erfüllt wird.
Die anderen Operatoren haben die bekannte Bedeutung.

Um einige Eigenschaften einfacher auszudrücken werden zusätzlich noch weitere Operatoren definiert \cite{clarke1982design}:
\begin{itemize}
	\item $\operatorname{EF}\varphi \equiv \EU{\top}{\varphi}$ und $\operatorname{AF}\varphi \equiv \AU{\top}{\varphi}$ bedeuten intuitiv, dass es einen Pfad gibt bzw. für alle Pfade irgendwann $\varphi$ gilt.
	\item $\operatorname{EG}\varphi \equiv \neg\operatorname{EF}\neg\varphi$ und $\operatorname{AG}\varphi \equiv \neg\operatorname{AF}\neg\varphi$ bedeuten, dass es einen Pfad gibt bzw. für alle Pfade gilt, dass in jedem Zustand $\varphi$ gilt.
\end{itemize}
Mithilfe dieser Syntax und zugehöriger Semantik auf Transitionssystemen wollen wir nun eine Beispielhafte $\CTL$-Formel betrachten:
\begin{example}[Beispiele für $\CTL$-Formeln und deren Bedeutung]
	Sei die Menge der atomaren Aussagen $\mathsf{AP}=\{\mathsf{idle}, \mathsf{error}\}$.
	Dann besagt die Formel $\operatorname{EF}\operatorname{AG}\mathsf{error}$, dass es einen Pfad gibt, auf dem ab irgendeinem Zustand alle Zustände die atomare Aussage $\mathsf{error}$ erfüllen. 
	Das heißt, es gibt einen Pfad mit einem irreversiblen Fehler.
\end{example}

Zusätzlich soll noch angemerkt werden, dass $0\in \mathbb{N}$ gilt.